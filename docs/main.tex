% ==========================================================
% iSSB-ΔTheory – Revision Paper (LaTeX master)
% ==========================================================
\documentclass[11pt,a4paper]{article}

% ---------- パッケージ ----------
\usepackage[utf8]{inputenc}          % 日本語を含むUTF-8
\usepackage[T1]{fontenc}
\usepackage{lmodern}                 % ベクターフォント
\usepackage{geometry}                % 余白設定
  \geometry{margin=25mm}
\usepackage{hyperref}                % 目次リンク
  \hypersetup{colorlinks=true, linkcolor=blue, citecolor=blue}
\usepackage{amsmath,amssymb,amsthm}  % 数式環境
\usepackage{bm}                      % 太字ベクトル
\usepackage{physics}                 % 物理コマンド
\usepackage{graphicx}                % 図
\usepackage{cite}                    % 参考文献番号をまとめる
\usepackage{authblk}                 % 複数著者
\usepackage{fancyhdr}                % ヘッダ/フッタ
\usepackage{siunitx}                 % 単位
\usepackage{enumitem}                % 箇条書きカスタム
\usepackage{fontspec}          % ←追加
% ---------- フォント ----------
\usepackage{fontspec}
%\usepackage{xeCJK}           % 日本語用
\setmainfont{TeX Gyre Pagella} % ラテン+ギリシャ (Δ もOK)
%\setCJKmainfont{IPAexMincho}  % 日本語 (どの文字も表示可)


%\DeclareUnicodeCharacter{0394}{\Delta} % ギリシャ大文字Δ

% ---------- ヘッダ/フッタ ----------
\pagestyle{fancy}
\fancyhf{}
\fancyhead[L]{iSSB-ΔTheory Revision Draft}
\fancyhead[R]{\thepage}

% ---------- 定理環境 ----------
\newtheorem{definition}{定義}
\newtheorem{theorem}{定理}
\newtheorem{lemma}{補題}

% ---------- 自作コマンド ----------
\newcommand{\dL}{\mathcal{L}}        % ラグランジアン
\newcommand{\Mpl}{M_{\mathrm{Pl}}}   % プランク質量
\newcommand{\Dphi}{\Delta\!\phi}     % Δφ

% ==========================================================
\begin{document}

\title{\bfseries iSSB-ΔTheory:\\
Dynamic Origin of Dark Energy and Unified Field Revision}
\author[1]{Mason Tabuchi\thanks{Corresponding author: \texttt{mason@example.com}}}
\author[2]{ChatGPT (o3)\thanks{AI Co-author}}
\affil[1]{Independent Researcher, Tokyo, Japan}
\affil[2]{OpenAI}
\date{\today}
\maketitle

\begin{abstract}
We present an updated formulation of the iSSB-ΔTheory, addressing the
main review comments regarding (i)~a fully covariant action principle,
(ii)~loop-level renormalisation of the dynamical vacuum term~$\kappa(H)$,
and (iii)~observational constraints from \textit{Planck} 2018 and BAO.
This draft compiles all ten original papers (I–X) into a single, coherent
framework suitable for peer-review submission.
\end{abstract}

\tableofcontents
\newpage

% ==========================================================
\section{Introduction}
\label{sec:intro}

ダークエネルギーの起源を**力学的に**説明する理論として  
iSSB-ΔTheory は\cite{Tabuchi2025_I,Tabuchi2025_X}  
以下の 3 つの柱を持つ:
\begin{enumerate}[label=\Roman*.]
  \item Δ場による自己対称性破れ(iSSB)  
  \item 非可換履歴時間 $\tau$ と情報密度
  \item トポロジカル保存則
\end{enumerate}
本稿では査読者からの主要指摘に応じ、  
\begin{itemize}
  \item \S\ref{sec:action} で Einstein–Hilbert + Δ場の\textbf{完全作用}を提示
  \item \S\ref{sec:loop} で $\kappa(H)$ の 1-loop 導出と数値評価
  \item \S\ref{sec:mcmc} で MCMC による\textbf{観測フィット}
\end{itemize}
を順に示す。

% ==========================================================
\section{Full Covariant Action}
\label{sec:action}

作用は
\begin{align}
  S &= \int d^4x \,\sqrt{-g}\,\Bigl[
        \frac{\Mpl^{2}}{2} R
      - \frac12 g^{\mu\nu}\partial_\mu \Phi \partial_\nu \Phi
      - V(\Phi)
      - \frac{\xi}{2} R \Phi^2
    \Bigr], \label{eq:action}
\end{align}
where $V(\Phi)=\frac{\lambda}{4}\Phi^4 + \frac{m^2}{2}\Phi^2$.
変分より…
% (後続で action/lagrangian.tex から \input する方法も可)

% ==========================================================
\section{One-loop Effective Vacuum Term}
\label{sec:loop}
% 簡潔な概要。詳細は loop/notebook.ipynb 参照。
%% placeholder
 % 無ければ後で削除
\providecommand{\PhiField}{\Phi}
% action/lagrangian.tex
\section{Full Covariant Action}\label{sec:action}

%\newcommand{\Mpl}{M_{\mathrm{Pl}}}
%\newcommand{\PhiField}{\Phi}

\begin{align}
S &= S_{\text{EH}} + S_{\Phi}, \\
S_{\text{EH}} &= \frac{\Mpl^{2}}{2}\int d^{4}x\sqrt{-g}\,R, \\
S_{\Phi} &= \int d^{4}x\sqrt{-g}\Bigl[
  -\frac12\,g^{\mu\nu}\partial_\mu\PhiField\,\partial_\nu\PhiField
  - V(\PhiField) - \frac{\xi}{2} R\PhiField^{2}
\Bigr], \\
V(\PhiField) &= \frac{\lambda}{4}\PhiField^{4} + \frac{m^{2}}{2}\PhiField^{2}.
\end{align}
% TODO: ξ の要否・多成分化など


% ==========================================================
\section{Observational Constraints}
\label{sec:mcmc}

Cobaya 設定ファイル \texttt{mcmc/config\_planck.yaml} を用い、
Planck2018 + BAO で
\[
  w(z) = w_0 + w_a (1-a)
\]
のパラメータ空間を探索した結果…

% ==========================================================
\section{Conclusion}
本稿では…

% ==========================================================
\bibliographystyle{unsrt}
% ----- 参考文献ファイルがまだ無い場合はコメントアウトしておいても PDF 出力可 -----
% \bibliography{ref}

\end{document}
